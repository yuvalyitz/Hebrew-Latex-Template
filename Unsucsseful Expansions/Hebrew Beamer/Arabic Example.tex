% compile with XeLaTeX
% this template was created by salim bou 
\documentclass[dvipsnames,mathserif]{beamer}
\usepackage{tikz}
\usetikzlibrary{calc}
\usepackage{polyglossia}
% \setdefaultlanguage[numerals=maghrib,locale=algeria]{hebrew}
% \setotherlanguage[numerals=maghrib,locale=algeria]{arabic}

\setdefaultlanguage[numerals=maghrib,locale=algeria]{arabic} % locale=mashriq, libya, algeria, tunisia, morocco, or mauritania  for names of months in \date 
\setotherlanguage{hebrew}
\setotherlanguage{english}
\newfontfamily\arabicfont[Script=Arabic]{Amiri}
\newfontfamily\arabicfontsf[Script=Arabic]{Amiri}

\usetheme{Warsaw}
%\usecolortheme{crane}

% for RTL liste
\makeatletter
\newcommand{\RTListe}{\raggedleft\rightskip\leftm}
\newcommand{\leftm}{\@totalleftmargin}
\makeatother



% RTL frame title
\setbeamertemplate{frametitle}
{\vspace*{-1mm}
  \nointerlineskip
    \begin{beamercolorbox}[sep=0.3cm,ht=2.2em,wd=\paperwidth]{frametitle}
        \vbox{}\vskip-2ex%
        \strut\hskip1ex\insertframetitle\strut
        \vskip-0.8ex%
    \end{beamercolorbox}
}


% align subsection in toc
\makeatletter
\setbeamertemplate{subsection in toc}
{\leavevmode\rightskip=5ex%
  \llap{\raise0.1ex\beamer@usesphere{subsection number projected}{bigsphere}\kern1ex}%
  \inserttocsubsection\par%
}
\makeatother

% RTL triangle for itemize
\setbeamertemplate{itemize item}{\scriptsize\raise1.25pt\hbox{\donotcoloroutermaths$\blacktriangleleft$}} 

%\setbeamertemplate{itemize item}{\rule{4pt}{4pt}}

\defbeamertemplate{enumerate item}{square2}
{\LR{
    %
    \hbox{%
    \usebeamerfont*{item projected}%
    \usebeamercolor[bg]{item projected}%
    \vrule width2.25ex height1.85ex depth.4ex%
    \hskip-2.25ex%
    \hbox to2.25ex{%
      \hfil%
      {\color{fg}\insertenumlabel}%
      \hfil}%
  }%
}}

\setbeamertemplate{enumerate item}[square2]

\setbeamertemplate{navigation symbols}{}

\begin{document}

\rightskip\rightmargin

\title{الهندسة الفضائية \\
\textcolor{yellow}{التعامد في الفضاء}}

\author{الأستاذ : سليم بو}
\institute{المستوى : سنة أولى جذع مشترك علوم وتكنولوجيا}
\date{\today}

\begin{frame}
\maketitle
\end{frame}

\begin{frame}{المحتويات}
\tableofcontents
\end{frame}



\section{تعامد المستقيمات في الفضاء}
\subsection{تعريف}
\begin{frame}{تعامد المستقيمات في الفضاء}

\begin{exampleblock}{تعريف}
نقول عن مستقيمين في الفضاء أنهما متعامدان إذا كان المستقيمان الموازيان لهما من نفس النقطة متعامدان
\end{exampleblock}

\pause

\textcolor{red}{مثال:} 
 
\begin{columns}

\begin{column}{0.4\linewidth}

\begin{tikzpicture}[scale=0.75]
\coordinate(A)at (0,1);
\coordinate(B)at (3,0);
\coordinate(C)at (3,2.5);
\coordinate(D)at (0,3.5);

\coordinate(E)at (2,2);
\coordinate(F)at (5,1);
\coordinate(G)at (5,3.5);
\coordinate(H)at (2,4.5);


\draw[thick](A)--(B)--(C)--(D)--cycle;
\draw[thick](B)--(F)--(G)--(H)--(D) (G)--(C);
\draw[dashed](H)--(E)--(F) (E)--(A);

\onslide<3->{\draw[red,very thick]($(H)!-0.3!(G)$)--($(G)!-0.3!(H)$)node[right]{\tiny $(D_2)$}; 
\draw[blue,very thick]($(B)!-0.3!(C)$)--($(C)!-0.3!(B)$)node[right]{\tiny $(D_1)$}; 
}

\onslide<4->{\fill[blue](G)circle(2.5pt)node[above right]{\tiny $A$};} 

\onslide<5->{\draw[dashed,blue,very thick]($(F)!-0.3!(G)$)--($(G)!-0.3!(F)$);
\draw[thick]($(G)!0.3cm!(H)$)--+(0,-0.3)--($(G)!0.3cm!(F)$);
}

\onslide<7->{\draw[OliveGreen,very thick]($(H)!-0.3!(D)$)--($(D)!-0.3!(H)$)node[above]{\tiny $(D_3)$}; }

\end{tikzpicture}

\end{column}

\begin{column}{0.5\linewidth}

\onslide<6->{

\begin{itemize}\RTListe
\item
من النقطة $A$ المستقيم الموازي لـ 
$(D_1)$ عمودي على المستقيم $(D_2)$ 
 فالمستقيمان $(D_1)$  و $(D_2)$ متعامدان. 
\item
نفس الشيء يمكن قوله عن  المستقيمين 
$(D_1)$ و $(D_3)$
\end{itemize}

\vspace{2cm}

}

\end{column}

\end{columns} 
 
\end{frame}

\subsection{خواص}

\begin{frame}{خواص}
\begin{exampleblock}{}
\begin{enumerate}\rightskip\leftm
\item 
المستقيم العمودي على أحد مستقيمين متوازيين عمودي على الآخر.

\item
المستقيمان الموازيان لمستقيمين متعامدين متعامدان.
\end{enumerate}
\end{exampleblock}

\pause

\begin{columns}

\begin{column}{0.3\linewidth}

\begin{tikzpicture}[scale=0.75]
\coordinate(A)at (0,0);
\coordinate(B)at (3,0);
\coordinate(C)at (3,3);
\coordinate(D)at (0,3);

\coordinate(E)at (1,1);
\coordinate(F)at (4,1);
\coordinate(G)at (4,4);
\coordinate(H)at (1,4);


\draw[thick](A)--(B)--(C)--(D)--cycle;
\draw[thick](B)--(F)--(G)--(H)--(D) (G)--(C);
\draw[dashed](H)--(E)--(F) (E)--(A);

\onslide<2->{\draw[red,very thick]($(H)!-0.3!(C)$)node[above]{\tiny $(\Delta_1)$}--($(C)!-0.3!(H)$);
\draw[red,very thick,dashed]($(E)!-0.3!(B)$)node[above]{\tiny $(\Delta_2)$}--($(B)!-0.3!(E)$);
\draw[red,very thick](B)--($(B)!-0.3!(E)$);
}

\onslide<3->{\draw[blue,very thick]($(G)!-0.2!(D)$)node[above]{\tiny $(D)$}--($(D)!-0.2!(G)$);
\draw[thick](2.2,3.4)--(2.4,3.45)--(2.2,3.55);
}

\end{tikzpicture}
\end{column}

\begin{column}{0.65\linewidth}


\begin{itemize}\RTListe

\item<2->
المستقيمان
 $(\Delta_1)$ و $(\Delta_2)$ متوازيان 

\item<3-> 
المستقيم 
$(D)$ عمودي على المستقيم $(\Delta_1)$

\item<4-> 
ومنه يكون المستقيم
$(D)$ عمودي على  $(\Delta_2)$ 

\end{itemize}
\end{column}
\end{columns}

\end{frame}
\section{تعامد المستقيمات والمستويات}
\subsection{تعريف}
\begin{frame}{تعامد المستقيمات والمستويات}
\pause
\begin{exampleblock}{تعريف}
نقول عن مستقيم أنه عمودي على مستو إذا كان هذا المستقيم عموديا على كلّ مستقيمات هذا المستوي.
\end{exampleblock}
 
\pause
 
\begin{block}{مبرهنة}
حتى يكون مستقيم عموديا على مستو يكفي أن يكون عموديا على مستقيمين متقاطعين من ذلك المستوي
\end{block} 
 
\pause 
 
 \textcolor{red}{مثال:}
 
\begin{columns} 
\begin{column}{0.35\textwidth}
\vspace{-3mm}
\centering

\begin{tikzpicture}[scale=0.5]
\coordinate(A)at (0,0);
\coordinate(B)at (3,0);
\coordinate(C)at (3,3);
\coordinate(D)at (0,3);

\coordinate(E)at (1,1);
\coordinate(F)at (4,1);
\coordinate(G)at (4,4);
\coordinate(H)at (1,4);


\draw[thick](A)--(B)--(C)--(D)--cycle;
\draw[thick](B)--(F)--(G)--(H)--(D) (G)--(C);
\draw[dashed](H)--(E)--(F) (E)--(A);

\onslide<2->{
\foreach \i in{B,F}{\fill[red](\i)circle(2.5pt)node[right,black]{\tiny $\i$};
}
\fill[red](A)circle(2.5pt)node[left,black]{\tiny $A$};
\draw[red,very thick]($(B)!-0.2!(C)$)--($(C)!-0.3!(B)$)node[below left]{\tiny $(\Delta)$};
\draw[thick](2.7,0)--(2.7,0.3)--(3,0.3);
\draw[thick](3,0.3)--(3.2,0.55)--(3.2,0.2);
}

\onslide<6->{\foreach \i in {0.2,0.4,...,2.8} {\draw($(E)+(\i,0)$)--($(A)+(\i,0)$);}}

\end{tikzpicture}
\end{column}
\begin{column}{0.6\textwidth}
\raggedleft
\only<5->{
المستقيم $(\Delta)$ عمودي على كل من المستقيمين : 
$(BA)$ و $(BF)$ ، أي أنه عمودي على المستوي 
$(ABF)$}

\end{column}
\end{columns}

\end{frame}


\subsection{خواص}
\begin{frame}{خواص}

\begin{exampleblock}{}
\begin{enumerate}\rightskip\leftm
\item<2-> 
يوجد مستقيم وحيد يشمل نقطة معلومة ويعامد مستو معلوما.

\item<2->
يوجد مستو وحيد يشمل نقطة معلومة ويعامد مستقيما معلوما.

\vspace{5mm}

\item<3->
المستويان العموديان على نفس المستقيم متوازيان.

\item<3->
المستقيمان العموديان على نفس المستوي متوازيان.

\vspace{5mm}

\item<4->
المستقيم العمودي على أحد مستويين متوازيين عمودي على الآخر.

\item<4->
المستوي العمودي على أحد مستقيمين متوازيين عمودي على الآخر.

\end{enumerate}
\end{exampleblock}

\end{frame}



\section{تعامد المستويات}
\subsection{تعريف}
\begin{frame}{تعامد المستويات}

\begin{exampleblock}{تعريف}
نقول عن مستويين أنهما متعامدان إذا شمل أحدهما  مستقيما عمودي على الآخر.
\end{exampleblock}

\pause

\textcolor{red}{مثال}

\begin{columns}

\begin{column}{0.35\linewidth}

\begin{tikzpicture}[scale=0.75]
\coordinate(A)at (0,1);
\coordinate(B)at (3,0);
\coordinate(C)at (3,2.5);
\coordinate(D)at (0,3.5);

\coordinate(E)at (2,2);
\coordinate(F)at (5,1);
\coordinate(G)at (5,3.5);
\coordinate(H)at (2,4.5);


\onslide<3>{\fill[red!30](C)--(B)--(F)--(G)--cycle;
\fill[green!30](C)--(G)--(H)--(D)--cycle;
}

\onslide<4->{\fill[red!30](C)--(B)--(F)--(G)--cycle;
\fill[gray!50](A)--(B)--(C)--(D)--cycle;
}

\draw[thick](A)--(B)--(C)--(D)--cycle;
\draw[thick](B)--(F)--(G)--(H)--(D) (G)--(C);
\draw[dashed](H)--(E)--(F) (E)--(A);


\foreach \i in{A,B,C,D,E,F,G,H}{\fill[blue](\i)circle(2pt)node[above left]{\tiny $\i$};}

\end{tikzpicture}

\end{column}

\begin{column}{0.6\linewidth}

\raggedleft

\onslide<2->{
ليكن متوازي المستطيلات 
$ABCDEFGH$
}

\begin{itemize}\RTListe

\item<3->
المستويان 
$(BCG)$ و $(DCG)$
 هما مستويان متعامدان 

 
\item<4->
نفس الشيء يمكن قوله عن  المستويين 
$(BCG)$ و $(ADC)$

\end{itemize}

\end{column}

\end{columns}

\end{frame}
\subsection{خواص}
\begin{frame}{خواص}
\begin{exampleblock}{}
\begin{enumerate}\rightskip\leftm

\item 
المستوي العمودي على أحد مستويين متوازيين عمودي على الآخر.

\item
	إذا كان $(P)$ و $(P')$ مستوين متقاطعين وكان كلّ
            منهما عموديا على مستو ثالث $(Q)$ فإنّ مستقيم 
            تقاطع المستويين $(P)$ و $(P')$ عمودي على المستوي $(Q)$.


\end{enumerate}
\end{exampleblock}
\end{frame}

\section{تطبيق}

\setbeamertemplate{enumerate item}[circle]

\begin{frame}{تمرين 47 صفحة 209}
$ABCD$ 
رباعي وجوه حيث $AD=DC$ و $AB=BC$ ،
$M$
منتصف $[AC]$.
\begin{enumerate}\RTListe

\item
بين أن المستقيم $(AC)$ عمودي على المستوي $(BDM)$.

\item
استنتج أن المستقيم $(AC)$ عمودي على المستقيم $(BD)$.

\end{enumerate}

\pause

\vspace{-5mm}

\begin{columns}
\begin{column}{0.27\textwidth}
\begin{tikzpicture}[scale=0.7,equal/.style={fill=red,inner sep=0pt,minimum width=2mm}]
\coordinate (D) at (0,0);
\coordinate (C) at (2.5,-1);
\coordinate (B) at (5,0);
\coordinate (A) at (2,4);
\coordinate (M) at ($(A)!0.5!(C)$);

\onslide<5->{\fill[red!20](M)--(D)--(B)--cycle;}

\draw [dashed](B)--(D);
\draw[thick](D)--(C)--(B)--(A)--cycle (A)--(C);

\foreach \i/\j in {A/above,B/right,C/below,D/left}{\fill[blue](\i)circle (2pt)node[\j]{\tiny $\i$};}

\onslide<4->{
\fill[red](M)circle(2pt)node[above right]{\tiny $M$};
\draw[red,thick](B)--(M)--(D);
\foreach \i in {A,C}{
\node[equal,rotate=5] at ($(M)!.5!(\i)$){};
}}
%
\onslide<3->{
\foreach \i in {.5,.52}{\node [equal,rotate=45] at ($ (A)!\i!(B) $) {};}
\foreach \i in {.5,.54}{\node [equal,rotate=100] at ($ (B)!\i!(C) $) {};}
\foreach \i in {.48,.5,.52}{\node [equal,rotate=-30] at ($ (A)!\i!(D) $) {};}
\foreach \i in {.47,.5,.53}{\node [equal,rotate=70] at ($ (D)!\i!(C) $) {};}
}
\end{tikzpicture}
\end{column}

\begin{column}{0.73\textwidth}

\begin{itemize}\RTListe
\item<6-| alert@6>
المثلثان
 $ABC$ و $ADC$
  متساويا الساقين، والمستقيمان 
  $(BM)$ و $(DM)$
   هما متوسطان فيهما وهما أيضا محوران .
   
   ومنه فالمستقيم  
   $(AC)$ عمودي على كل منهما 
  أي أنه عمودي على المستوي الذي يشملهما  
  $(BDM)$.

\item<7-| alert@7>
المستقيم 
$(AC)$ عمودي على المستوي 
$(BDM)$ فهو عمودي على كل مستقيماته أي أنه عمودي على المستقيم 
$(BD)$.

\end{itemize}

\onslide<8->{}

\end{column}

\end{columns}

\end{frame}

\end{document}









