\documentclass{article} 


% hyperlinks
\usepackage{amsmath}
\usepackage{hyperref}
\usepackage{cleveref}

\usepackage{algorithm}
\usepackage[noend]{algpseudocode}

\usepackage{multicol}
\usepackage{verbatim}
\usepackage{graphicx}

\usepackage{polyglossia} %This is the package that gives access to fonts.
% for theorems
\usepackage{amsthm}
\usepackage{ amssymb }

\usepackage{tikz}
\usepackage{tikz-network}
\usetikzlibrary{arrows.meta}
\usetikzlibrary{matrix}
\usetikzlibrary{positioning,chains,fit,shapes,calc}
\usetikzlibrary{calc,intersections,through,hobby}
\usetikzlibrary{mindmap}
\usetikzlibrary {angles,quotes} 

\usepackage{subcaption}
\usetikzlibrary{trees}

\usepackage{pgfplots}
\pgfplotsset{compat=1.18} 


% % Define the Hebrew theorem environment
\newtheorem{theorem}{משפט}
\newtheorem{deinition}{הגדרה}
% \crefname{theorem}{משפט}{משפטים}
% \crefname{deinition}{הגדרה}{הגדרות}
\newtheorem{claim}{טענה}
% \crefname{claim}{טענה}{טענות}


% \usepackage[T1]{fontenc}

% Used to include part of tex files in others
\usepackage{catchfilebetweentags}
% A FUCKING BRILLIANT Patch to fix issue of catchfilebetweentags: it removes newlines...
\makeatletter
\def\CatchFBT@Fin@l#1[#2]{%
   \begingroup
      %\endlinechar\m@ne % <- this is the guilty party
      \makeatletter #2%
      \scantokens\expandafter{%
         \expandafter\CatchFBT@tok\expandafter{\the\CatchFBT@tok}}%
      \CatchFBT@IsAToken{#1}
         {\global#1\expandafter{\the\CatchFBT@tok}}
         {\xdef#1{\the\CatchFBT@tok}}%
      \ifx\CatchFBT@tok#1\else\global\CatchFBT@tok{}\fi
   \endgroup
}% \CatchFBT@Final
\makeatother


% good af for maths
\usepackage{amsmath}

% Change bullet types in itemize
\usepackage{enumitem}
\usepackage{pifont}

% Headers
\usepackage[headheight=12mm,headsep=3mm]{geometry}
\usepackage{fancyhdr}
\pagestyle{fancy}

% Table
\usepackage{amsfonts}
\usepackage{booktabs}
\usepackage{siunitx}

% These two packages
\usepackage{etoolbox}
\usepackage{fontspec}
\setmainfont{Linux Libertine O}
\newfontfamily\quotefont{Cardo}
\AtBeginEnvironment{quote}{\quotefont}


\setdefaultlanguage{hebrew}  %this makes the titles RTL, switch these if you write mostly in English
\setotherlanguage{english}  
\newfontfamily\hebrewfont[Script=Hebrew]{Miriam Mono CLM} %This font will work in Overleaf, Linux and Mac installs, if you're running windows you'd need to pick the right ones
\newfontfamily\hebrewfontsf[Script=Hebrew]{Miriam CLM}
% \newfontfamily\hebrewfontrm[Script=Hebrew]{Cardo}

% \setsansfont[Script=Hebrew]{Arial}
% \setmonofont{Arial}

\usepackage{tabularx,booktabs,multirow} % all about tables: booktabs gives the standard formatting, multirow allows cells to span over multiple rows.



 \newcommand{\problemdef}[3]{
  \begin{center}
    \begin{minipage}{0.7\textwidth}
      \noindent
      \fontspec{Miriam Mono CLM}
      #1

      \vspace{2pt}
      \setlength{\tabcolsep}{3pt}
      \begin{tabularx}{\textwidth}{lX lX}
        \textbf{קלט:} 	& #2 \\
        \textbf{שאלה:} 	& #3
      \end{tabularx}
    \end{minipage}
  \end{center}
}


% removes footnote line separator
\renewcommand*\footnoterule{}
% footnote in alphabeth
\renewcommand{\thefootnote}{\alph{footnote}}

\newcommand{\NP}{\textenglish{\textnormal{\textsf{NP}}}}
\newcommand{\PP}{\textenglish{\textnormal{\textsf{P}}}}
% \newcommand{\NP}{NP}
% \newcommand{\PP}{P}
\newcommand{\bigO}{\mathcal{O}}

\usepackage[realmainfile]{currfile}

% Compare Strings
\newcommand{\StrCompare}[2]{%
  \ifthenelse{\equal{\string \detokenize{#1}}{\string \detokenize{#2}}}
    {TRUE}
    {FALSE}%
}{}

% substrings
\usepackage{stringstrings}

% get unit number
\newcommand{\documentNumber}[0]{\substring{\currfilebase}{3}{4}\par}

% colors
\definecolor{LightCyan}{rgb}{0.88,1,1}




% \newtheorem{theorem}{\protect\theorem}
% \addto\captionsenglish{%
%   \renewcommand{\theorem}{Theorem}%
% }
% \addto\captionshebrew{%
%   \renewcommand{\theorem}{\textbf{הגדרה}}%
% }



% CREATE RANDOM INTS
% \usepackage{pgf}
% % \pgfmathsetseed{\number\randomseed} % to ensure that it is randomized 
% \newcommand{\randomnum}[1]{% 
% \pgfmathsetmacro{\a}{1}%
% \pgfmathsetmacro{\b}{int(#1)}% 
% \pgfmathsetmacro{\thenum}{int(random(\a,\b))}%
% \thenum%
% }%
% \randomnum{10}


% Champions Tip
\usepackage{stackengine}
\usepackage{scalerel}
\usepackage{xcolor,amssymb}
\newcommand\champTip[1][2ex]{%
  \scaleto{
  \stackengine{-0.1pt}{\color{white}\tiny\bfseries !}{\scalebox{1.}[.9]{%
  \color{red}$\blacktriangle$}}{O}{c}{F}{F}{L}
  }{#1}%
}



%<*SETUP>

\title{ 
\textbf{
כותרת המסמך
}

מסמך גנרי}
\author{ }
\date{}


% \lhead{\textenglish{Latex for Mehadrin}}
% \rhead{
% שכוייח
% }
\pagestyle{fancy}% Change page style to fancy
\fancyhf{}% Clear header/footer
\lhead[L]{\texthebrew{שבלונה\\
יחידה \documentNumber (מספר תלוי בשם הקובץ)
}}
\fancyhead[R]{\includegraphics[height=7.5mm]{logo.png}}
\fancyfoot[C]{\thepage}

\usepackage[T1]{fontenc}
\usepackage[utf8]{inputenc}


\begin{document}
\maketitle
\renewcommand{\thefootnote}{\alph{footnote}}

אני מניח שמי שפתח את השבלונה הזאת לא צריך הסברים למה לאטך זה להיט\footnote{
אם צריך אנסה להראות במסמך זה את כל הדברים הנחמדים שאפשר לעשות איתו.
}.

\paragraph{מה אפשר לעשות?}

\begin{enumerate}
    \item להשתמש בנוסחאות כמו
    $f(x)= 3x$
    בתוך הטקסט.
    \item 
    קצת לוקח זמן להתרגל לזה.
    \item 
    כי צריך להקפיד שהמתמטיקה בשורה נפרדת מהמלל בעברית.
    \begin{enumerate}
        \item אפשר לבחור לעשות אנומרציה בעברית
         \item נחמד לא?
    \end{enumerate}
    \item תסתכלו גם על השבלונות
    \textit{AA00.tex}
    ו
    \textit{AA00S.tex}.
    אפשר לערוך בעזרתם קבצי מטלות ופתרונות בקלות.
    \item
    לשלב אנגלית עם עברית כמו פה
(\text{Three Two One}),
רק שעדיין לא טרחתי לסדר את הפונטים..
אז אני הופך את הסדר של המילים אם יש יותר ממילה אחת, לרוב אין.

\item ליצור סביבות ממוספרות להגדרות ומשפטים ורפרנסים כמו ל
הגדרה
\ref{def:cooldef}.


\end{enumerate}

אפשר להכניס קישורים לטקסט כמו זה:
\href{https://www.youtube.com/watch?v=YX40hbAHx3s}{\textbf{בחיים אל תלחצו על לינק שאתם לא סומכים עליו}} 

\paragraph{מה (עדיין) אי אפשר לעשות?}

\begin{enumerate}
    \item להציג מקורות מימין לשמאל \cite{greenwade93}.
    % \item להשתמש בסביבות ממוספרות להגדרות ומשפטים.
    \item להעביר את הקוד למצגת בקלות \footnote{יש פרוייקט נפרד שמשתמש במהדר אחר...}.
\end{enumerate}



\section{איורים גרפים אלגוריתמים וכו׳}

ניתן לשלב בטקסט הגדרות, משפטים וטענות כגון:

\begin{deinition}
\label{def:cooldef}
גרף הוא זוג סדור 
$G=(V,E)$ 
כש
$E\subseteq V\times V$.
\end{deinition}

ניתן לראות דוגמא לגרף באיור
\ref{fig:G}.

\begin{claim}
\label{thm:coolclaim}
כל הגרפים הם מגניבים.
\end{claim}
\begin{proof}
    נניח בשלילה שקיים גרף לא מגניב - כבר אנחנו רואים שההנחה שגויה כי איך גרף יהיה לא מגניב?
\end{proof}


\begin{theorem}
\label{thm:coolthm}
טענה \ref{thm:coolclaim}
משתמשת בהגיון מעגלי.
\end{theorem}
\begin{proof}
העיקר הבנו את הנקודה של הסביבות נכון?
\end{proof}

\begin{theorem}
\label{thm:1}
כל קבוצה של סביבות (משפטים, הגדרות או טענות) ממוספרת בנפרד.
\end{theorem}



\subsection{גרפים וכאלה}
\subsubsection{גרפים וכאלה}
\subsubsection{גרפים וכאלה}
\subsection{גרפים וכאלה}

\begin{figure}[H]
    \centering
    \begin{tikzpicture}[scale=1.2]
\begin{axis}[
% grid=both,
            ticks=none,
          xmax=100,ymax=10,
          % ymode=log,
          yticklabels={,,},
          xticklabels={,,},
          axis lines=middle,
          enlargelimits,
          xlabel={input length $n$},
          xlabel near ticks,
          y label style={at={(-0.1,1.)}},
          ylabel={time}
          ]


\addplot[smooth,black,mark=none,%samples=140,
line width=1.5pt,domain=4:100,
samples=50,
color=violet]  {30/x} node[left] {}; 
\node[] at (104,.75) {\color{violet} $\frac{1}{n}$};

\addplot[smooth,black,mark=none,%samples=140,
line width=1.5pt,domain=1:100,
samples=13,
color=pink]  {1.5} node[above left] {$c$};          


\addplot[smooth,black,mark=none,%samples=140,
line width=1.5pt,domain=1:100,
samples=13,
color=magenta]  {ln(x)/2+.5} node[above left] {$\log n$};

\addplot[smooth,black,mark=none,%samples=140,
line width=1.5pt,domain=0:100,
samples=63,
color=blue]  {sqrt(x)/2} node[above left] {$\sqrt{n}$};

\addplot[smooth,black,mark=none,%samples=140,
line width=1.5pt,domain=0:100,
samples=13,
color=cyan]  {x*0.075} node[above left] {$n$};

\addplot[smooth,black,mark=none,%samples=140,
line width=1.5pt,domain=0:90,
samples=13,
color=black!20!green]  {x*ln(x)*0.022} node[above left] {$n \log n$};

\addplot[smooth,black,mark=none,%samples=140,
line width=1.5pt,domain=0:70,
samples=10,
color=black!60!green]  {x^2*0.0022} node[above left] {$n^2$};
\node[] at (60,10) {\color{black!60!green} $n^2$};

\addplot[smooth,black,mark=none,%samples=140,
line width=1.5pt,domain=0:46,
samples=10,
color=black!60!pink]  {x^3*0.00014} node[above left] {$n^3$};
\node[] at (35,10) {\color{black!60!pink} $n^3$};

\addplot[smooth,black,mark=none,%samples=140,
line width=1.5pt,domain=0:24,
samples=12,
color=orange]  {2^x*0.000001} node[below left] {$2^n$};

\node[] at (18,10) {\color{orange} $2^n$};

\addplot[smooth,black,mark=none,%samples=140,
line width=1.5pt,domain=0:9,
samples=15,
color=red]  {x^x*0.0000001} node[below left] {$n!$};

\node[] at (4,10) {\color{red} $n!$};

\end{axis}
\end{tikzpicture}
    \caption{ מי לא אוהב גרפים צבעוניים בטיקז?}
    \label{fig:enter-label}
\end{figure}


אפשר ליצור טבלאות גם יחסית בקלות.

\begin{center}
\renewcommand{\arraystretch}{1.75}
\begin{tabular}{ | c | c | c | }
\hline
 \textbf{סימון} &\textbf{ הגדרה 1 }&\textbf{ הגדרה 2} \\ 
 \hline
 $f\in \bigO(g)$
 &
 $ \lim\limits_{n\to\infty}\ \frac{f(n)}{g(n) } < \infty$
 &
 % $\exists c,n_0 \ :\ \forall n>n_0\ :\ $
 $  f(n) \leq c \cdot g(n) $
 \\ 
 % % \hline

 $f\in o(g)$
 &
 $ \lim\limits_{n\to\infty}\ \frac{f(n)}{g(n) }= 0$
 &
 % $\exists c,n_0 \ :\ \forall n>n_0\ :\ $
 $ f(n) < c \cdot g(n)$
 \\ 
 % \hline

 $f\in \Omega(g)$
 &
 $ \lim\limits_{n\to\infty}\ \frac{g(n)}{f(n) } < \infty $
 &
 % $\exists c,n_0 \ :\ \forall n>n_0\ :\ $
 $ f(n) \geq c \cdot g(n)$
\\
 % \hline
  $f\in \omega(g)$
 &
 $ \lim\limits_{n\to\infty}\ \frac{g(n)}{f(n) }= 0$
 &
 % $\exists c,n_0 \ :\ \forall n>n_0\ :\ $
 $  f(n) > c \cdot g(n)$\\
 % \hline
  $f\in \Theta(g)$
 &
 $ f\in \bigO(g)$ וגם $ g\in \bigO(f)$ 
 &
 % $\exists c,n_0 \ :\ \forall n>n_0\ :\ $
 $ c\cdot g(n) \leq f(n) \leq c' \cdot g(n)$\\
\hline
\end{tabular}
\end{center}


\begin{figure}[H]
    \centering
    \begin{tikzpicture}
    \begin{axis}[
            ybar interval,
            xmajorgrids=false,
            ticks=none,
          yticklabels={,,},
          xticklabels={,,},
          axis lines=middle,
          enlargelimits,
          xlabel={איטרציות $m$},
          xmax=10,ymax=55,
          xlabel near ticks,
          y label style={rotate=90, at={(-0.,.75)}},
          ylabel={לאיטרציה זמן},
    ]
        \addplot[fill=orange] plot coordinates
            {(0,15) (1,50) (2,30) (3,20) (4,40) (5,10)
            (6,30) (7,20) (8,40) (9,30)};

    \addplot[smooth,ultra thick, dashed, red] plot coordinates {
        (0,50)
        (9.25,50)
    };
    \node[] at (9.,55) {\color{red} $\bigO(f(n))$};
    
    \end{axis}
    \end{tikzpicture}
    \caption{
    בגדול זה לא סרט להתאים את הטיקז שלכם לעברית.
    }
    \label{fig:enter-label}
\end{figure}



    \begin{figure}[H]
    \centering
    \begin{subfigure}{.3\linewidth}
    \scalebox{0.4}{
    \begin{tikzpicture}
    
    \draw[ultra thick,fill=gray!20] (1,1) rectangle (10,10);
    
    \foreach \v\p/\q [
    evaluate=\p as \pplus using \numexpr\p+1\relax
    % remember=\p as \lastp (initially 0)
    ] in {
        1/1/{2, 6, 9},  %1 A
        2/2/{6, 5, 7},  %2 B
        3/3/{4,8,7, 9, 5, 6},  %3 C
        4/4/{5, 6, 7, 8},  %4 D
        5/5/{9},  %5 E
        6/6/{8},     %6 F
        7/7/{2},  %7 G
        8/8/{4},       %8 H
        9/9/{1}         %9 I
    }{
    \node at (\p+.5,10.5) (\p1) [] {\huge$\v$};
    \node at (0.5,10-\p+.5) (\p2) [] {\huge$\v$};
    \foreach \r 
    [
    evaluate=\r as \rplus using \numexpr\r+1\relax,
    evaluate=\r as \rminus using \numexpr\r\relax
    ]
        in \q { 
            \draw[ultra thick,fill=gray!75] (\rminus,11-\p) rectangle (\rplus,11-\pplus) {};
            % For undirected graphs uncomment the following
            \draw[ultra thick,fill=gray!75] (\p,11-\rminus) rectangle (\pplus,11-\rplus) {};
            
        }
    }

    \foreach \v\p/\q [
    evaluate=\p as \pplus using \numexpr\p+1\relax
    % remember=\p as \lastp (initially 0)
    ] in {
        % 1/1/{2, 6, 9},  %1 A
        % 2/2/{6, 5, 7},  %2 B
        3/3/{4,8, 6},  %3 C
        4/4/{6, 8},  %4 D
        % 5/5/{9},  %5 E
        6/6/{8},     %6 F
        % 7/7/{2},  %7 G
        % 8/8/{4},       %8 H
        % 9/9/{1}         %9 I
    }{
    \node at (\p+.5,10.5) (\p1) [] {\huge$\v$};
    \node at (0.5,10-\p+.5) (\p2) [] {\huge$\v$};
    \foreach \r 
    [
    evaluate=\r as \rplus using \numexpr\r+1\relax,
    evaluate=\r as \rminus using \numexpr\r\relax
    ]
        in \q { 
            \draw[ultra thick,fill=black!75] (\rminus,11-\p) rectangle (\rplus,11-\pplus) {};
            % For undirected graphs uncomment the following
            \draw[ultra thick,fill=black!75] (\p,11-\rminus) rectangle (\pplus,11-\rplus) {};
            
        }
    }
    
    \end{tikzpicture} 
}
    \end{subfigure}~\begin{subfigure}{.3\linewidth}
\begin{tikzpicture}
        \begin{scope}[every node/.style={circle,thick,draw}]
\foreach \i in {1,2,5,7,9} {
    \pgfmathsetmacro\r{(\i+2)*(360/9)}
    \node (\i) at (\r:1.75) {\i};
}
\foreach \i in {3,4,6,8} {
    \pgfmathsetmacro\r{(\i+2)*(360/9)}
    \node[fill=yellow!50] (\i) at (\r:1.75) {\i};
}

\end{scope}

\begin{scope}[>={Stealth[black]},
              every node/.style={circle},
              every edge/.style={draw=blue,very thick}]

    \path [-] (1) edge[]  node {} (2);
    \path [-] (1) edge[]  node {} (9);
    \path [-] (1) edge[]  node {} (6);

    \path [-] (2) edge[]  node {} (6);
    \path [-] (2) edge[]  node {} (7);
    \path [-] (2) edge[]  node {} (5);

    \path [-] (3) edge[]  node {} (9);
    \path [-] (3) edge[]  node {} (7);
    \path [-] (3) edge[]  node {} (4);
    \path [-] (3) edge[]  node {} (6);
    \path [-] (3) edge[]  node {} (8);
    \path [-] (3) edge[]  node {} (5);

    \path [-] (4) edge[]  node {} (5);
    \path [-] (4) edge[]  node {} (6);
    \path [-] (4) edge[]  node {} (8);
    \path [-] (4) edge[]  node {} (7);

    \path [-] (5) edge[]  node {} (9);
    
    \path [-] (6) edge[]  node {} (8);
    
    
\end{scope}
\end{tikzpicture}
    \end{subfigure}
 \caption{
\texthebrew{
הקוד בטיקז הזה מייצר גרפים ומטריצת שכנויות בו זמנית.
}
 }   
\label{fig:G}
\end{figure}

אפשר גם להוסיף אלגוריתמים אם משתמשים בסביבה דוברת אנגלית כדי להקיף את הקוד שלהם.

\vspace{-.5cm}
\textenglish{
\begin{center}
\begin{minipage}{.5\linewidth}%
\begin{algorithm}[H]
\caption*{\text{SuperCoolAlgorithm$(x,y)$}}\label{alg:countsort}
\begin{algorithmic}[1]
\State Take $x$. \Comment{\parbox[t]{.4\linewidth}{\texthebrew{ הערות שוליים}}}
\State Set $i = 3$
\While{ $i\leq n$} 
    \State  Wear sunscreen.
    \State  $i\leftarrow i+1$.
\EndWhile 
\Return $y$
\end{algorithmic}
\end{algorithm}
\end{minipage}
\end{center}
}

\bibliographystyle{plain}

\textenglish{
\bibliography{sample}
}
\end{document}

